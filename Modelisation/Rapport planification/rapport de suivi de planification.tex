\documentclass[]{article}

% Use utf-8 encoding for foreign characters
\usepackage[utf8]{inputenc}

% Setup for fullpage use
\usepackage{fullpage}
\usepackage{graphicx}

\usepackage[francais]{babel}

\usepackage{times}
%\usepackage{rotate}
%\usepackage{lscape}

\usepackage{color}

\usepackage{needspace}

\usepackage{float}

\newcommand{\placeholder}[1]{{\noindent \color{red}[ #1 ]}}

\begin{document}

\title{
{\Huge Rapport de suivi de planification}\\
Projet de Génie Logiciel\\
\smallskip
{\small Activité d'Apprentissage \textsf{S-INFO-015}}\\
}

\author{Groupe numéro: 3\\
Membres du groupe:\\
\textbf{DOM Eduardo , DHEUR Victor, AMEZIAN Aziz}\\
}


\date{Année Académique : 2017 - 2018\\
BAC 2 en Sciences Informatiques\\
\vspace{1cm}
Faculté des Sciences, Université de Mons}

\maketitle              % typeset the title of the contribution

\bigskip
\begin{center} \today \end{center}

\newpage
%%%%%%%%%%%%%%%%%%%%%%%%%%%%%%%%%%%%%%%%%%%%%%%%
%%%%%%%%%%%%%%%%%%%%%%%%%%%%%%%%%%%%%%%%%%%%%%%%
\section{Retards}\label{sec:retards}
A cette étape du projet, nous n'avons subi aucun retard par rapport aux delais spécifiés dans notre rapport de planification.
\section{Risques}\label{sec:risques}
Nous n'avons trouvé aucun risque supplémentaire durant notre modélisation.
\section{Ressources}\label{sec:res}
Nous nous sommes aperçus qu'une librairie XML serait utile pour l'implémentation. Par contre, nous avons abandonné l'utilisation de Yakindu. Le reste des ressources reste inchangé.
\section{Ordonnancement}\label{sec:ord}
Nous avons suivi l'ordonnancement prévu pour la modélisation. L'ordonnancement de l'implémentation nous paraît toujours correct.
\end{document}
